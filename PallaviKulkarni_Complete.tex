%%%%%%%%%%%%%%%%%%%%%%%%%%%%%%%%%%%%%%%%%%%%%%%%%%%%%%%%%%%%%%%%%%%%%%%%
%%%%%%%%%%%%%%%%%%%%%% Simple LaTeX CV Template %%%%%%%%%%%%%%%%%%%%%%%%
%%%%%%%%%%%%%%%%%%%%%%%%%%%%%%%%%%%%%%%%%%%%%%%%%%%%%%%%%%%%%%%%%%%%%%%%

%%%%%%%%%%%%%%%%%%%%%%%%%%%%%%%%%%%%%%%%%%%%%%%%%%%%%%%%%%%%%%%%%%%%%%%%
%% NOTE: If you find that it says                                     %%
%%                                                                    %%
%%                           1 of ??                                  %%
%%                                                                    %%
%% at the bottom of your first page, this means that the AUX file     %%
%% was not available when you ran LaTeX on this source. Simply RERUN  %%
%% LaTeX to get the ``??'' replaced with the number of the last page  %%
%% of the document. The AUX file will be generated on the first run   %%
%% of LaTeX and used on the second run to fill in all of the          %%
%% references.                                                        %%
%%%%%%%%%%%%%%%%%%%%%%%%%%%%%%%%%%%%%%%%%%%%%%%%%%%%%%%%%%%%%%%%%%%%%%%%

%%%%%%%%%%%%%%%%%%%%%%%%%%%% Document Setup %%%%%%%%%%%%%%%%%%%%%%%%%%%%

% Don't like 10pt? Try 11pt or 12pt
\documentclass[10pt]{article}

% This is a helpful package that puts math inside length specifications
\usepackage{calc}


% Simpler bibsection for CV sections
% (thanks to natbib for inspiration)
\makeatletter
\newlength{\bibhang}
\setlength{\bibhang}{1em}
\newlength{\bibsep}
 {\@listi \global\bibsep\itemsep \global\advance\bibsep by\parsep}
\newenvironment{bibsection}%
        {\vspace{-\baselineskip}\begin{list}{}{%
       \setlength{\leftmargin}{\bibhang}%
       \setlength{\itemindent}{-\leftmargin}%
       \setlength{\itemsep}{\bibsep}%
       \setlength{\parsep}{\z@}%
        \setlength{\partopsep}{0pt}%
        \setlength{\topsep}{0pt}}}
        {\end{list}\vspace{-.6\baselineskip}}
\makeatother

% Layout: Puts the section titles on left side of page
\reversemarginpar

%
%         PAPER SIZE, PAGE NUMBER, AND DOCUMENT LAYOUT NOTES:
%
% The next \usepackage line changes the layout for CV style section
% headings as marginal notes. It also sets up the paper size as either
% letter or A4. By default, letter was used. If A4 paper is desired,
% comment out the letterpaper lines and uncomment the a4paper lines.
%
% As you can see, the margin widths and section title widths can be
% easily adjusted.
%
% ALSO: Notice that the includefoot option can be commented OUT in order
% to put the PAGE NUMBER *IN* the bottom margin. This will make the
% effective text area larger.
%
% IF YOU WISH TO REMOVE THE ``of LASTPAGE'' next to each page number,
% see the note about the +LP and -LP lines below. Comment out the +LP
% and uncomment the -LP.
%
% IF YOU WISH TO REMOVE PAGE NUMBERS, be sure that the includefoot line
% is uncommented and ALSO uncomment the \pagestyle{empty} a few lines
% below.
%

%% use the package to refer to URL's
\usepackage{url}

%% Use these lines for letter-sized paper
%\usepackage[paper=letterpaper,
            %includefoot, % Uncomment to put page number above margin
%            marginparwidth=1.2in,     % Length of section titles
%            marginparsep=.05in,       % Space between titles and text
%            margin=1in,               % 1 inch margins
%            includemp]{geometry}

%% Use these lines for A4-sized paper
\usepackage[paper=a4paper,
%            %includefoot, % Uncomment to put page number above margin
            marginparwidth=30.5mm,    % Length of section titles
            marginparsep=1.5mm,       % Space between titles and text
            margin=25mm,              % 25mm margins
            includemp]{geometry}

%% More layout: Get rid of indenting throughout entire document
\setlength{\parindent}{0in}

%% This gives us fun enumeration environments. compactitem will be nice.
\usepackage{paralist}

%% Reference the last page in the page number
%
% NOTE: comment the +LP line and uncomment the -LP line to have page
%       numbers without the ``of ##'' last page reference)
%
% NOTE: uncomment the \pagestyle{empty} line to get rid of all page
%       numbers (make sure includefoot is commented out above)
%
\usepackage{fancyhdr,lastpage}
\pagestyle{fancy}
%\pagestyle{empty}      % Uncomment this to get rid of page numbers
\fancyhf{}\renewcommand{\headrulewidth}{0pt}
\fancyfootoffset{\marginparsep+\marginparwidth}
\newlength{\footpageshift}
\setlength{\footpageshift}
          {0.5\textwidth+0.5\marginparsep+0.5\marginparwidth-2in}
\lfoot{\hspace{\footpageshift}%
       \parbox{12.7in}{\, \hfill %
                    \begin{flushleft}Pallavi K (updated Mar 2022) \end{flushleft} \vspace*{-20pt} \hfill{\arabic{page}} 
                    %of \protect\pageref*{LastPage}}  % +LP
                    %\arabic{page}                               % -LP
                    \hfill \,}}

% Finally, give us PDF bookmarks
\usepackage{color,hyperref}
\definecolor{darkblue}{rgb}{0.0,0.0,0.3}
\hypersetup{colorlinks,breaklinks,
            linkcolor=darkblue,urlcolor=darkblue,
            anchorcolor=darkblue,citecolor=darkblue}

%%%%%%%%%%%%%%%%%%%%%%%% End Document Setup %%%%%%%%%%%%%%%%%%%%%%%%%%%%


%%%%%%%%%%%%%%%%%%%%%%%%%%% Helper Commands %%%%%%%%%%%%%%%%%%%%%%%%%%%%

% The title (name) with a horizontal rule under it
% (optional argument typesets an object right-justified across from name
%  as well)
%
% Usage: \makeheading{name}
%        OR
%        \makeheading[right_object]{name}
%
% Place at top of document. It should be the first thing.
% If ``right_object'' is provided in the square-braced optional
% argument, it will be right justified on the same line as ``name'' at
% the top of the CV. For example:
%
%       \makeheading[\emph{Curriculum vitae}]{Your Name}
%
% will put an emphasized ``Curriculum vitae'' at the top of the document
% as a title. Likewise, a picture could be included:
%
%   \makeheading[\includegraphics[height=1.5in]{my_picutre}]{Your Name}
%
% the picture will be flush right across from the name.
\newcommand{\makeheading}[2][]%
        {\hspace*{-\marginparsep minus \marginparwidth}%
         \begin{minipage}[t]{\textwidth+\marginparwidth+\marginparsep}%
             {\huge \mdseries \begin{center} #2 \end{center} \hfill #1}%
                 \rule{\columnwidth+0.1in}{1pt}%
	\end{minipage}}
\newcommand{\makesubheading}[2][]%
        {\hspace*{-\marginparsep minus \marginparwidth}%
	 \begin{minipage}[t]{\textwidth+\marginparwidth+\marginparsep}%
             {\mdseries \begin{center} #2 \end{center} \hfill #1}%
	\end{minipage}}

% The section headings
%
% Usage: \section{section name}
%
% Follow this section IMMEDIATELY with the first line of the section
% text. Do not put whitespace in between. That is, do this:
%
%       \section{My Information}
%       Here is my information.
%
% and NOT this:
%
%       \section{My Information}
%
%       Here is my information.
%
% Otherwise the top of the section header will not line up with the top
% of the section. Of course, using a single comment character (%) on
% empty lines allows for the function of the first example with the
% readability of the second example.
\renewcommand{\section}[2]%
        {\pagebreak[3]\vspace{1.3\baselineskip}%
         \phantomsection\addcontentsline{toc}{section}{#1}%
         \hspace{0in}%
         \marginpar{
         \raggedright \scshape #1}#2}

% An itemize-style list with lots of space between items
\newenvironment{outerlist}[1][\enskip\textbullet]%
        {\begin{itemize}[#1]}{\end{itemize}%
         \vspace{-.6\baselineskip}}

% An environment IDENTICAL to outerlist that has better pre-list spacing
% when used as the first thing in a \section
\newenvironment{lonelist}[1][\enskip\textbullet]%
        {\vspace{-\baselineskip}\begin{list}{#1}{%
        \setlength{\partopsep}{0pt}%
        \setlength{\topsep}{0pt}}}
        {\end{list}\vspace{-.6\baselineskip}}

% An itemize-style list with little space between items
\newenvironment{innerlist}[1][\enskip\textbullet]%
        {\begin{compactitem}[#1]}{\end{compactitem}}

% An environment IDENTICAL to innerlist that has better pre-list spacing
% when used as the first thing in a \section
\newenvironment{loneinnerlist}[1][\enskip\textbullet]%
        {\vspace{-\baselineskip}\begin{compactitem}[#1]}
        {\end{compactitem}\vspace{-.6\baselineskip}}

% To add some paragraph space between lines.
% This also tells LaTeX to preferably break a page on one of these gaps
% if there is a needed pagebreak nearby.
\newcommand{\blankline}{\quad\pagebreak[3]}
\newcommand{\halfblankline}{\quad\vspace{-0.5\baselineskip}\pagebreak[3]}

% Uses hyperref to link DOI
\newcommand\doilink[1]{\href{http://dx.doi.org/#1}{#1}}
\newcommand\doi[1]{doi:\doilink{#1}}

% For \url{SOME_URL}, links SOME_URL to the url SOME_URL
\providecommand*\url[1]{\href{#1}{#1}}
% Same as above, but pretty-prints SOME_URL in teletype fixed-width font
\renewcommand*\url[1]{\href{#1}{\texttt{#1}}}

% For \email{ADDRESS}, links ADDRESS to the url mailto:ADDRESS
\providecommand*\email[1]{\href{mailto:#1}{#1}}
% Same as above, but pretty-prints ADDRESS in teletype fixed-width font
%\renewcommand*\email[1]{\href{mailto:#1}{\texttt{#1}}}

%\providecommand\BibTeX{{\rm B\kern-.05em{\sc i\kern-.025em b}\kern-.08em
%    T\kern-.1667em\lower.7ex\hbox{E}\kern-.125emX}}
%\providecommand\BibTeX{{\rm B\kern-.05em{\sc i\kern-.025em b}\kern-.08em
%    \TeX}}
\providecommand\BibTeX{{B\kern-.05em{\sc i\kern-.025em b}\kern-.08em
    \TeX}}
\providecommand\Matlab{\textsc{Matlab}}

%%%%%%%%%%%%%%%%%%%%%%%% End Helper Commands %%%%%%%%%%%%%%%%%%%%%%%%%%%

%%%%%%%%%%%%%%%%%%%%%%%%% Begin CV Document %%%%%%%%%%%%%%%%%%%%%%%%%%%%

\begin{document}
\makeheading{ Pallavi Kulkarni \\ \vspace*{12pt} 
\normalsize \href{https://www.sfsu.edu/}{Department of Embedded Electrical \& Computer Systems}, \normalsize \href{https://www.sfsu.edu/}{San Francisco State University}\\ 
\normalsize (+91)9740911633 $|$ \email{pallavikulkarni1@mail.sfsu.edu}\\
\normalsize \emph {\href {https://www.linkedin.com/in/pallavi-kulkarni-aug2009/} {LinkedIn Profile}} $|$ \emph {\href {https://github.com/PallaviK-Git} {Github Profile}}}\\ 

\vspace*{2pt} 
\section{Education}
\vspace*{-10pt} \begin{innerlist} \item \href{https://www.sfsu.edu/}{San Francisco State University}, San Francisco, CA. USA \hfill{Jan 2021 - {Present}} \\ 
\textbf{M.S.}, \href{https://engineering.sfsu.edu/master-science-electrical-and-computer-engineering}{\textbf{Embedded Electrical and Computer Systems}} \hfill{GPA: 3.0/4.0} \\
Project: \emph{TBD} \\  
\end{innerlist}

\vspace*{2pt} \begin{innerlist} \item \href{http://www.vtu.ac.in/}{Visvesvaraih Technological University}, Belgaum, India \hfill{Sept 2001 - Feb '08} \\ 
\textbf{B.E.}, {\textbf{Electronics \& Communications Engineering}} %\hfill{GPA: /100}
\end{innerlist}
\vspace*{2pt}


\section{Graduate\\ Coursework}
%
\begin{tabular}[t]{p{3in} l}
Hardware for Machine Learning & Neural Machine Interface\\
Embedded Systems & Advanced Digital Design\\
Advanced Computer Communication \& Networks & Computer Systems\\
\end{tabular}

%\halfblankline
\vspace*{10pt}


\section{Graduate \\ Course \\ Projects}

\vspace*{-14pt}
\textbf{Comparison of Machine Learning Methods on EMG data of Hand Gestures using Matlab} \\ 
\vspace{4pt}
Course: \textbf{Neural Machine Interface} \\
\vspace*{-34pt} \begin{flushright} Jan 2021 - May 2021 \end{flushright}
\vspace*{-5pt}
\textbf{Platform, tools and languages:} \emph{Windows, Matlab Machine Learning toolbox, Matlab code} 
\vspace*{2pt}
\begin{innerlist} \item Aim of the project was to compare the performance of the Machine learning methods on the Electromyographic (EMG) data of hand gestures. Data from seven gestures that were repeated for six times each by one subject was obtained using 8 channel Myo armband at 200 Hz data-rate for an average of 1.1s. Four time-domain features including mean absolute value (MAV), zero crossing (ZC), sign slope change (SSC) or turn count (TC) and Willison amplitude value (WAV) were computed from the raw EMG data for each channel. Four classification models including Linear Discriminant Analysis (LDA), Naive Bayes, Support Vector Machine (SVM), and K-Nearest Neighbor (KNN) were used for training and testing. Finally, training accuracy  and testing accuracy of the four methods were compared. KNN produced the best accuracy of 98.4\% during training while SVM produced the best accuracy of 96.06\% during testing. Overall, SVM produced the best accuracy of 95.1\% and 96.06\% during both training and testing respectively.
\end{innerlist}  \vspace*{10pt}

\textbf{Tiva Network Tester} \\ 
\vspace{2pt}
Course: \textbf{Embedded Systems} \\
\vspace*{-34pt} \begin{flushright} Aug 2021 - Dec 2021 \end{flushright}
\vspace*{-5pt}
\textbf{Platform, tools and languages:} \emph{Windows, Code Composer Studio, ENC28J60 driver module, TivaWare, Embedded C}
\vspace*{2pt}
\begin{innerlist} \item Developed of a basic network stack and shell for the Tiva C-series microcontrollers using Embedded C. Microcontroller was interfaced with a Ethernet Breakout board to connect to a router on one end and a laptop-computer on another end to view and interact with the shell . Following commands were implemented and made available from the shell:
   \subitem \emph{ping} - Pings and IPv4 address.
   \subitem \emph{arp} - Looks up the IPv4 address of a MAC address.
   \subitem \emph{raw} - Toggles raw printing of Ethernet frames.
   \subitem \emph{ipconfig} - Displays IPv4 configuration.
   \subitem \emph{setip} - Sets IPv4 address.
   \subitem \emph{setsub} - Sets IPv4 subnet.
   \subitem \emph{setgw} - Sets IPv4 gateway.
   \subitem \emph{help} - Prints available comands.
   \subitem \emph{uptime} - Reports time that Tiva has been on the network.
\end{innerlist}  \vspace*{2pt}

\textbf{ASIC Implementation of Motion Estimator in 32/28nm CMOS} \\ 
\vspace{2pt}
Course: \textbf{Advanced Digital Design} \\
\vspace*{-34pt} \begin{flushright} Aug 2021 - Dec 2021 \end{flushright}
\vspace*{-5pt}
\textbf{Platform, tools and languages:} \emph{Linux, MobaXterm,Bash script, Verilog, Synopsis Verilog Compiler Simulator, PrimeTime tool and ICC2}
\vspace*{2pt}
\begin{innerlist} \item Motion estimation is one of the key components of high compression video codecs. One of the most popular algorithms for motion estimation is block matching algorithm due to its simplicity in implementation on hardware. The goal of this project was to design and implement and end-to-end motion estimator including High Description Langugue (HDL) design, timing and area simulation, redesign to improve timing and area, and finally generate the General Description Scheme (GDS-II) file that could subsequently be used to print on reticle/photomask for integrated circuit (IC) chip fabrication targeting 32/28 nm CMOS technology.
\end{innerlist}  \vspace*{4pt}
%\vspace*{2pt}


\section{Skills}
%
\begin{tabular}[t]{l  p{3.1in}}
\emph{Platforms (Desktop and Server)}  &  Windows, Linux(Ubuntu, Redhat, Fedora, CentOS, AWS Linux, Kali, Solaris 11).\\
\emph{Programming Languages}  & C, MATLAB, Python, PHP, HTML, Golang, UNIX shell scripting.\\
\emph{Databases and tools}  & MySQL, SQLyog, phpMyAdmin.\\
\emph{Linux based tools}  &  gdb, gcc, gprof, valgrind make, disk formatting (fdisk, cfdisk), System monitoring tools (top, atop, SAR), rsync, cron, minicom (serial communication), arm-gcc (ARM Cross compiler), SVN, Git (Github, Bitbucket).\\
\emph{Assembly Languages}  &  Microprocessors:8085, 8086; Microcontrollers:8051, ARM M3.\\
\emph{IDE and other tools}  & VS Code, Eclipse, CTags, \LaTeX{}.\\
\emph{Web server Environment}  & (X/W)AMP, LAMP, Apache-Tomcat, Moodle based LMS, IIS.\\
\end{tabular}
\vspace*{2pt}

\section{Area of\\ Interests}
\begin{loneinnerlist}
\item Machine Learning
\item Embedded Systems Security
\item IoT Security
\end{loneinnerlist}
\vspace*{4pt}


\section{Professional \\ Work \\Experience}
\textbf{Consultant, Bangalore} \\ \textbf{Developer \& Server Administrator} \\                      
\vspace*{-34pt} \begin{flushright} {July 2016 - Present} \end{flushright}
\vspace*{-5pt} 
\textbf{Platforms, tools and languages:} \emph{Java, PHP, Git, Oracle Virtualbox, Windows RDP, Vmware, Putty}
\vspace*{2pt}
\begin{innerlist} \item Implement Software based process automations, upgradations and inventory.
\subitem	PHP based applications (auto mailing module, portal for invoice bidding).
\subitem    Implemented Macros on MS Excel for automating tasks.
\item	Successfully led transitioning of a banking application to the Datacenter
\subitem	\textbf{\emph{Hardware included following servers:}} RHEL based CISCO UCS 240 M4, Solaris based Oracle T7-1 and Oracle Super cluster M7-8.
\subitem	Audited the implementation of all the above mentioned category of servers, resolved the implementation risks, and finally trained the sustenance team for regular monitoring of the servers.
\item	Setup and maintain hardware, cloud, and software infrastructure for Banking customer
\subitem	\textbf{\emph{OS and configuration:}} LAMP and Apache-Tomcat server, Windows based Apache-Tomcat servers for Java and PHP based Web application development and deployment on Test and Production servers in-house, and cloud environments (GoDaddy).
\subitem	Additional roles include backup, restoration, and database management \\

\end{innerlist}  \vspace*{14pt}

\textbf{ASM Technologies, Bangalore} \\ \textbf{Learning Facilitator and Software developer} \\                      
\vspace*{-34pt} \begin{flushright} {Aug 2015 - July '16} \end{flushright}
\vspace*{-1pt} 
\textbf{Platforms, tools and languages:} \emph{Java, MySQL, Git}
\vspace*{2pt}
\begin{innerlist}
\item \textbf{Training:} Single Point of Contact(SPoC) for Embedded training deliveries, handled training sessions on Linux basics commands and Shell scripting for embedded development, and Infra Management Service Batches.
\item	Besides training, I set up Ubuntu based LAMP and Apache-Tomcat servers for Web application deployment on test and production servers, and cloud environments (inlcuidng AWS, Linode, Go4Hosting, and Netmagic)
\item	An additional development role was to automation of Financial Office System by using Java, and MYSQL to implement Taxation and Auditing software and tools.
\subitem	Specifically, the role included updating Database on the Production server, coordinating with the customer, Field trials, Release Management (including the entire process of installation on Server, deploying the application, and managing it on both the test and production servers).
\end{innerlist}  \vspace*{8pt}

\textbf{Vimarshana Technology Solutions Pvt. Ltd., Bangalore} \\ \textbf{Learning Facilitator} \\ 
\vspace{-35pt} \begin{flushright} {Aug 2011 - Aug '15} \end{flushright}
\textbf{Assistant Learning Facilitator} \\
\vspace*{-36pt} \begin{flushright} {Jan 2010 -    Jul '11} \end{flushright}
\textbf{Research Associate} \\
\vspace*{-35pt} \begin{flushright} {Aug 2009 -   Jan '10} \end{flushright}
\vspace*{2pt}

\vspace*{-5pt} 
\textbf{Languages and tools:} \emph{Matlab, LabVIEW, Java, PHP, MySQL, SVN}
\vspace*{2pt}

\begin{innerlist}
\item \textbf{Projects and management:}
\subitem	\emph {Computer vision project} -- Pattern Matching by Correlating Pixels of an Image, implemented algorithm on Matlab, and simulated a prototype product using LabVIEW’s Image processing toolbox.
\subitem	\emph {Lifescapes (PHP, MYSQL based Real Estate)} – Versioning, Testing, Customer coordination, Field trials, Release Management, Server setup on AWS and Application deployment.
\subitem	\emph {Subscriber Management Billing System (PHP, MYSQL based Telecom Billing System)} – Versioning, Testing, Design documentation, Customer coordination, Field trials, Release Management.
\subitem	\emph {CRM (Java, MYSQL based Process Management)} – Versioning, Release Management, Application deployment on Production Server and Database migration.
\item \textbf{Training:}
\subitem	I was the lead trainer and learning facilitator for various training deliveries, including trainings on Golang, Linux basics commands and Shell scripting for development, HTML with PHP, C, L2/L3 Telecom Protocol testing and Infra Management Service Batches.
\subitem	Offered training on Linux kernel porting to the Devkit8000 (OMAP3 -- ARM Cortex-A8+C64x) board, and Embedded Boot-loaders (xloader, U-boot).
\subitem	\emph {CDAC Training Coordinator} – managing the Training, Feedback, Assessment, Project Evaluation and Placement processes.
 \\
%\textbf{Tools and languages:} \emph{Matlab, LabVIEW, Java, PHP, MySQL, SVN}
\end{innerlist} 
% \vspace*{2pt}


\section{Awards \\ and\\ Scholarships}

\begin{loneinnerlist}
\item[1.] Full scholarship from Governments of Karnataka (state level) and India (federal level) towards my middle and high school education.
%\vspace*{-26pt} \\ \begin{flushright}1994-1999 \end{flushright}
\end{loneinnerlist}\vspace*{10pt}


\section{Certifications \\ and\\ Activities}
\vspace*{14pt}
\begin{loneinnerlist}
\item[1.] \vspace*{-14pt}
\textbf{SFSU Hackathon 2022} \\ 
\vspace{2pt} 
Title: \textbf{Serotonin Booster} \\
\vspace*{-30pt} \begin{flushright} Mar 11-13 2022 \end{flushright}
\vspace{10pt} 
\vspace*{-5pt}
\textbf{Platform, tools and languages:} \emph{Windows, VS Code, Python, DeepFace, OpenCV, Flask}
\vspace*{2pt}
\begin{innerlist} \item We built the Model on Python using OpenCV and Deepface. OpenCV was used to detect the face and Deepface was used to analyse the facial expression to check what is the current mood of the person. The whole model was put onto a web app using Flask.
\end{innerlist}  \vspace*{10pt}
\item[2.]Presented a technical seminar on \textbf{“MEMS Micro-manipulators”}. \vspace*{-7pt} \\
\item[3.] \textbf{Advanced Diploma Course} in Real time operating systems (\textbf{RTOS}) from Cranes International Ltd. \vspace*{-7pt} \\
\item[4.] Certification in \textbf{Information Systems Management} from APTECH on Webpage Deign and Publishing using FP 2000. \vspace*{-7pt} \\
\end{loneinnerlist}

\end{document}

%%%%%%%%%%%%%%%%%%%%%%%%%% End CV Document %%%%%%%%%%%%%%%%%%%%%%%%%%%%%

%----------------------------------------------------------------------%
% The following is copyright and licensing information for
% redistribution of this LaTeX source code; it also includes a liability
% statement. If this source code is not being redistributed to others,
% it may be omitted. It has no effect on the function of the above code.
%----------------------------------------------------------------------%
% Copyright (c) 2007, 2008, 2009, 2010, 2011 by Theodore P. Pavlic
%
% Unless otherwise expressly stated, this work is licensed under the
% Creative Commons Attribution-Noncommercial 3.0 United States License. To
% view a copy of this license, visit
% http://creativecommons.org/licenses/by-nc/3.0/us/ or send a letter to
% Creative Commons, 171 Second Street, Suite 300, San Francisco,
% California, 94105, USA.
%
% THE SOFTWARE IS PROVIDED "AS IS", WITHOUT WARRANTY OF ANY KIND, EXPRESS
% OR IMPLIED, INCLUDING BUT NOT LIMITED TO THE WARRANTIES OF
% MERCHANTABILITY, FITNESS FOR A PARTICULAR PURPOSE AND NONINFRINGEMENT.
% IN NO EVENT SHALL THE AUTHORS OR COPYRIGHT HOLDERS BE LIABLE FOR ANY
% CLAIM, DAMAGES OR OTHER LIABILITY, WHETHER IN AN ACTION OF CONTRACT,
% TORT OR OTHERWISE, ARISING FROM, OUT OF OR IN CONNECTION WITH THE
% SOFTWARE OR THE USE OR OTHER DEALINGS IN THE SOFTWARE.
%----------------------------------------------------------------------%